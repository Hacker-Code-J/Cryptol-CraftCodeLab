In Cryptol `\texttt{foldl}`  is a higher-order function that reduces a sequence (or list) to a single value by iteratively applying a binary function, starting from the left side of the sequence. It is similar to the fold operation found in many functional programming languages.

To understand `\texttt{foldl}`, lets break it down mathematically. Given: \begin{itemize}
	\item A binary operation `\texttt{f}' of type `\texttt{(b, a) -> b}'
	\item An initial value `\texttt{z}' of type `\texttt{b}'
	\item A sequence `\texttt{xs}' of type `\texttt{[n]a}' (a sequence of `\texttt{n}' elements, each of type `\texttt{a}')
\end{itemize}
The `\texttt{foldl}' function can be defined as: \[
foldl\ f\ z\ [x_0,x_1,\dots,x_{n-1}]
\]

This can be described recursively as:
\begin{enumerate}
	\item If the sequence is empty, the result is the initial value `\texttt{z}'.
	\item Otherwise, apply the function `\texttt{f}' to the initial value `\texttt{z}' and the first element of the sequence `\texttt{x\_0}', then recursively apply `\texttt{foldl}' to the result of this function application and the rest of the sequence.
\end{enumerate}
Mathematically, this is: \begin{align*}
	&foldl\ f\ z\ []\ = z \\
	&foldl\ f\ z\ (x:xs)\ = foldl\ f\ (f\ z\ x)\ xs \\
\end{align*}

\textbf{Example}
Let's take a concrete example to illustrate foldl. Suppose we want to compute the sum of a list of numbers using foldl.

Let:
\begin{itemize}
	\item $f(a, b) = a + b$ (binary addition function)
	\item $z = 0$ (initial value)
	\item $xs = [1, 2, 3, 4]$ (sequence of numbers)
\end{itemize}
Using foldl to compute the sum:
\[
foldl (+) 0 [1,2,3,4]
\]
Step-by-step:
\begin{enumerate}
	\item Start with initial value $z = 0$.
	\item Apply the function to the initial value and the first element:
	$f(0,1)=0+1=1$
	\item Apply foldl to the result and the rest of the sequence:
	$foldl (+) 1 [2,3,4]$
	\item Repeat:
	$f(1,2)=1+2=3$,
	$foldl (+) 3 [3,4]$
	\item Continue:
	$f(3,3)=3+3=6$,
	$foldl (+) 6 [4]$
	\item Finally:
	$f(6,4)=6+4=10$,
	$foldl (+) 10 [ ]$
\end{enumerate}
Since the sequence is now empty, the result is 10.

\textbf{Summary}
\begin{itemize}
	\item `\texttt{foldl}' starts with an initial value and iterates through the sequence from left to right.
	\item It applies a binary function to the current accumulated value and the current element of the sequence.
	\item The result of this function application becomes the new accumulated value.
	\item The process repeats until the sequence is exhausted, at which point the accumulated value is returned as the result.
\end{itemize}
Understanding foldl helps in performing various reduction operations over sequences in a concise and functional manner in Cryptol.