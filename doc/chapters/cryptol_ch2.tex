\chapter{Functional Cryptography and Software Verification with Cryptol}

\section{Introduction to Cryptol}

\textbf{Cryptol} is a purely functional programming language for cryptographic specifications. It's got a large amount of Haskell influence.
\begin{itemize}
	\item Very syntactically similar to Haskell
	\item And a similar type system (with some additions, and reductions)
	\item Purely functional
	\item Clearly the second best programming language by these metrics alone.
\end{itemize}

Some additions.
\begin{itemize}
	\item Type level arithmetic, and a solver for it
	\item Verification technology
	\item A smattering of other small scoping 
\end{itemize}

Some reductions.
\begin{itemize}
	\item It has type classes, but you can't write them.
	\item No built in IO or any other operations.
	\item No user-defined data type or ``pattern matching'' beyond the built-in sequence type.
\end{itemize}

\newpage
\noindent\textbf{C:}
\begin{lstlisting}[style=C]
uint32_t f(uint32_t x, uint32_t y) {
	return x + y;
}
\end{lstlisting}
\vspace{8pt}
\textbf{Haskell:}
\begin{lstlisting}[style=haskell]
f :: Word32 -> Word32 -> Word32
f x y = x + y
\end{lstlisting}

\noindent\textbf{Cryptol:}
\begin{lstlisting}[style=haskell]
f : [32] -> [32] -> [32]
f x y = x + y
\end{lstlisting}

Large or ``weird'' numbers.
\begin{lstlisting}[style=haskell]
f : [17] -> [17]
f x = -x

g : [23424] -> [548683776]
g x = x * x
\end{lstlisting}

Sequence are fundamental
\begin{lstlisting}[style=haskell]
[0 .. 3] = [0, 1, 2, 3]
f : { x } x -> [4]x
f x = [x, x, x, x]
// Define a sequence of 5 integers
sequenceOfInts : [5]Integer
sequenceOfInts = [1, 2, 3, 4, 5]

// Define a sequence of bits (binary sequence)
sequenceOfBits : [8]Bit
sequenceOfBits = [1, 0, 1, 1, 0, 0, 1, 0]

// Define a nested sequence (matrix of bits)
matrixOfBits : [3][4]Bit
matrixOfBits = [[1, 0, 1, 0], [0, 1, 0, 1], [1, 1, 1, 1]]

// Accessing elements and slices
firstElement = sequenceOfInts @ 0      // Access the first element (index starts from 0)
firstTwoBits = sequenceOfBits @ [0..1] // Access the first two bits
secondRow = matrixOfBits @ 1           // Access the second row of the matrix

// Sequence operations
reversedSequence = reverse sequenceOfInts    // Reverse the sequence
concatenatedSequence = sequenceOfInts # [6] // Concatenate [6] to the end of the sequence
\end{lstlisting}

\newpage
\begin{lstlisting}[style=haskell]
// Define a sequence of 16 bits
myBits : [16]Bit
myBits = [1, 0, 1, 1, 0, 0, 1, 0, 1, 1, 1, 0, 1, 0, 0, 1]

// Split the sequence into chunks of 4 bits
splitChunks : [4][4]Bit
splitChunks = split(myBits)

// Define a list of sequences
myChunks : [4][4]Bit
myChunks = [[1, 0, 1, 1], [0, 0, 1, 0], [1, 1, 1, 0], [1, 0, 0, 1]]

// Join the chunks into a single sequence
joinedSequence : [16]Bit
joinedSequence = join(myChunks)
\end{lstlisting}

\newpage
\section{Split and Join}
\begin{lstlisting}[style=C]
// Pseudo-C code representing what 'g' might look like in C
void g(type *z, type (*result)[a], int a) {
	// Assuming 'z' is an array of length 2*a and 'type' is whatever 'b' represents
	// 'result' is an array of two elements, each of which can hold 'a' items of 'type'
	for (int i = 0; i < a; i++) {
		result[0][i] = z[i];      // Copy the first half of 'z' into 'result[0]'
		result[1][i] = z[i + a];  // Copy the second half of 'z' into 'result[1]'
	}
}
\end{lstlisting}
In this pseudo-C version, g takes an array z and splits it into two halves, storing these in result[0] and result[1]. This is analogous to the Cryptol version where split takes z and returns two sequences, y and x, which get assigned in reverse order due to the pattern {y, x} in the Cryptol code.

\begin{align*}
&\texttt{// Split into four bytes (8-bit)}\\
&\texttt{split 0xAABBCCDD : [4][8] = {0xAA, 0xBB, 0xCC, 0xDD}} \\&\texttt{// Concatenation} \\
&\texttt{join [0xAA, 0xBB, 0xCC, 0xDD] = 0xAABBCCDD}
\end{align*}
\begin{lstlisting}[style=haskell]
split : { parts, each, a}
	  (fin each
	  ) => [parts * each]a -> [parts][each]a

g: (a, b) (fin a) => [2 * a]b -> [2][a]b
g z = {x, y}
	where {y, x} = split z
\end{lstlisting}

\newpage
\section{Verification}

\textbf{The Quarter Round - C99}

\begin{lstlisting}[style=C]
static void
qround(uint32_t* x, // Pointer to an array
	   uint32_t a, // Index of 'x', representing 'a'
	   uint32_t b, // Second index
	   uint32_t c, // ...
	   uint32_t d, // ...
) {
	/* L32 is left 32-bit ROLL, not a shift */
	x[a] += x[b]; x[d] = L32(x[d] ^ x[a], 16);
	x[c] += x[d]; x[b] = L32(x[b] ^ x[c], 12);
	x[a] += x[b]; x[d] = L32(x[d] ^ x[a],  0);
	x[c] += x[d]; x[b] = L32(x[b] ^ x[c],  7);
}
\end{lstlisting}

Cryptol has a sister tool, called \textbf{SAW}, which can do this using deep black magic that we dare not speak of (yet).

SAW has its own typed language - called SAWScript - used for scripting proofs in an automated way, which we'll use.

SAW doesn't work on surface-level syntax. It works with intermediate representations, often ones your compiler splits out.
In particular, SAW comes with tools for ingesting \textbf{LLVM bitcode, JVM Bytecode and Cryptol programs}.