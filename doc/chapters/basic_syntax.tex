\chapter{CH1}

\section{Basic Syntax}

\subsection*{Identifiers}

\begin{table}[h!]
\begin{center}
	\textit{Examples of Identifier}
\end{center}
{\ttfamily
\begin{tabular*}{\textwidth}{@{\extracolsep{\fill}}|llll|}
\hline
\textcolor{blue}{name} & name1 & name' & longer\_name \\
\textcolor{violet}{Name} & Name2 & Name'' & longerName\\
\hline
\end{tabular*}}
\end{table}

\subsection*{Keywords and Built-in Operators}

\begin{table}[h!]
\begin{center}
\textit{Keywords}
\end{center}
{\ttfamily
\begin{tabular*}{\textwidth}{@{\extracolsep{\fill}}|lllllll|}
\hline
as         &     extern  &    include &     interface &     parameter &     property  &    where \\
by         &     hiding  &    infix   &     let       &     pragma    &     submodule &    else \\
constraint &     if      &    infixl  &     module    &     primitive &     then & \\
down       &     import  &    infixr  &     newtype   &     private   &     type & \\
\hline
\end{tabular*}}
\end{table}

\newpage
\subsection*{Built-in Type-level Operators}

\begin{table}[h!]
\begin{center}
\textit{Keywords}
\end{center}\centering
\begin{tabular}{|c|c|}\hline
\normalfont\textbf{Operator} & \normalfont\textbf{Meaning} \\
\midrule
$+$ & Addition \\
$-$ & Subtraction \\
$*$ & Multiplication \\
$/$ & Division \\
$\slash\textasciicircum$ & Ceiling Division ($/$ rounded up) \\
$\%$ & Modulus \\
$\%\textasciicircum$ & Ceiling Modulus (Computing Padding) \\
$\textasciicircum\textasciicircum$ & Exponentiation \\
\texttt{lg2} & Ceiling logarithm (base 2) \\
\texttt{width} & Bit Width (equal to \texttt{lg2(n+1)})  \\
\texttt{max} & Maximum \\
\texttt{min} & Minimum \\
\hline
\end{tabular}
\end{table}

\subsection*{Numeric Literals}
\begin{table}[h!]
\begin{center}
\textit{Examples of Literals}
\end{center}
{\ttfamily\begin{tabular}{ll}
254 & \textcolor{green!50!black}{// Decimal literal} \\
0254 & \textcolor{green!50!black}{// Decimal literal} \\
0b11111110 & \textcolor{green!50!black}{// Binary literal} \\
0xFE & \textcolor{green!50!black}{// Hexadecimal literal} \\
0xfe & \textcolor{green!50!black}{// Hexadecimal literal} \\
\end{tabular}}
\end{table}

\begin{table}[h!]
\begin{center}
\textit{Polynomial Literals}
\end{center}
{\ttfamily\begin{tabular}{ll}
<| x$\textasciicircum\textasciicircum$6 + x$\textasciicircum\textasciicircum$4 + x$\textasciicircum\textasciicircum$2 + x$\textasciicircum\textasciicircum$1 + 1 |> & \textcolor{green!50!black}{// : [7], equal to 0b1010111} \\
<| x$\textasciicircum\textasciicircum$4 + x$\textasciicircum\textasciicircum$3 + x |>  & \textcolor{green!50!black}{// : [5], equal to 0b11010}
\end{tabular}}
\end{table}

\begin{table}[h!]
\begin{center}
\textit{Fractional Literals}
\end{center}
{\ttfamily\begin{tabular}{ll}
10.2 & \\
10.2e3 & \textcolor{green!50!black}{// 10.2 * 10$\textasciicircum$3} \\
0x30.1  & \textcolor{green!50!black}{// 3 * 64 + 1/16} \\
0x30.1p4  & \textcolor{green!50!black}{// (3 * 64 + 1/16) * 2$\textasciicircum$4}
\end{tabular}}
\end{table}

\begin{table}[h!]
\begin{center}
\textit{Using $\_$}
\end{center}
{\ttfamily\begin{tabular*}{\textwidth}{@{\extracolsep{\fill}}|ll|}
\hline
0b\_0000\_0010 &\\
0x\_FFFF\_FFEA &\\
\hline
\end{tabular*}}
\end{table}

\newpage
\section{Expressions}

\subsection*{Calling Functions}
\begin{table}[h!]
{\ttfamily\begin{tabular}{ll}
\textcolor{blue}{f} 2 & \textcolor{green!50!black}{// call `f' with parameter `2'} \\
\textcolor{blue}{g} x y & \textcolor{green!50!black}{// call `g' with two parameters: `x' and `y'} \\
\textcolor{blue}{h} (x,y) & \textcolor{green!50!black}{// call `h' with one parameter,  the pair `(x,y)'} \\
\end{tabular}}
\end{table}

