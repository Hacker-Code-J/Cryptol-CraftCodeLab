\section*{Cryptol vs EasyCrypt}
\begin{itemize}
\item \textbf{Cryptol}: Cryptol is a domain-specific language designed specifically for specifying cryptographic algorithms. A creation by Galois, Inc., it's a tool used primarily for creating high-assurance cryptographic software. Cryptol allows developers to write cryptographic algorithms in a way that directly reflects the mathematical specifications, which makes it easier to analyze and verify for correctness and security.
\item \textbf{EasyCrypt}: On the other hand, EasyCrypt is a toolset designed for the formal verification of cryptographic proofs. It provides a framework for developing and verifying mathematical proofs of the security of cryptographic constructions, such as encryption schemes, signature schemes, and hash functions. EasyCrypt operates at a higher level of abstraction compared to Cryptol and is used for proving the security properties of cryptographic protocols mathematically.
\end{itemize}

If you're comparing them from a user perspective, Cryptol is more about the implementation and specification of cryptographic algorithms, making sure they are implemented correctly according to their mathematical definitions. EasyCrypt is more about proving the theoretical security properties of cryptographic protocols and systems.